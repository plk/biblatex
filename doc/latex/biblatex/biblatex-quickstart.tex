\documentclass{ltxdockit}[2011/03/25]
\usepackage{btxdockit}

\newcommand*{\biber}{\sty{biber}\xspace}
\newcommand*{\biblatex}{\sty{biblatex}\xspace}
\newcommand*{\biblatexml}{\sty{biblatexml}\xspace}
\newcommand*{\biblatexhome}{http://sourceforge.net/projects/biblatex/}
\newcommand*{\biblatexctan}{%
  http://www.ctan.org/tex-archive/macros/latex/contrib/biblatex/}
\usepackage{hyperref}
\usepackage{zref-xr}
\usepackage{zref-user}
\zexternaldocument*{biblatex}

\usepackage{fontspec}
\setmonofont{Courier New}
\setmainfont[Ligatures=TeX]{Linux Libertine O}
\setsansfont[Ligatures=TeX]{Linux Biolinum O}
\usepackage[american]{babel}
\usepackage[strict]{csquotes}
\usepackage{tabularx}
\usepackage{longtable}
\usepackage{booktabs}
\usepackage{shortvrb}
\usepackage{microtype}
\usepackage{typearea}
\usepackage{mdframed}
\usepackage{graphicx}
\usepackage[flushmargin]{footmisc}
\areaset[current]{370pt}{700pt}
\lstset{
    basicstyle=\ttfamily,
    commentstyle=\color{red}\ttfamily,
    keepspaces=true,
    upquote=true,
    frame=single,
    breaklines=true,
    postbreak=\raisebox{0ex}[0ex][0ex]{\ensuremath{\color{red}\hookrightarrow\space}}
}
\KOMAoptions{numbers=noenddot}
\addtokomafont{title}{\sffamily}
\addtokomafont{paragraph}{\spotcolor}
\addtokomafont{section}{\spotcolor}
\addtokomafont{subsection}{\spotcolor}
\addtokomafont{subsubsection}{\spotcolor}
\addtokomafont{descriptionlabel}{\spotcolor}
\setkomafont{caption}{\bfseries\sffamily\spotcolor}
\setkomafont{captionlabel}{\bfseries\sffamily\spotcolor}
\pretocmd{\cmd}{\sloppy}{}{}
\pretocmd{\bibfield}{\sloppy}{}{}
\pretocmd{\bibtype}{\sloppy}{}{}
\makeatletter
\patchcmd{\paragraph}
  {3.25ex \@plus1ex \@minus.2ex}{-3.25ex\@plus -1ex \@minus -.2ex}{}{}
\patchcmd{\paragraph}{-1em}{1.5ex \@plus .2ex}{}{}
\makeatother

\MakeAutoQuote{«}{»}
\MakeAutoQuote*{<}{>}
\MakeShortVerb{\|}

\newrobustcmd*{\msecpref}[1]{%
  \leavevmode\marginpar{\msecref{#1}}}

\newcommand*{\msecref}[1]{B\zref{#1}}

\titlepage{%
  title={\biblatex Quickstart Guide},
  subtitle={},
  url={\biblatexhome},
  author={Philip Kime},
  email={},
  revision={1.0},
  date={\today}}

\hypersetup{%
  pdftitle={\biblatex Quickstart Guide},
  pdfsubject={Programmable Bibliographies and Citations},
  pdfauthor={Philip Kime},
  pdfkeywords={tex, latex, bibtex, bibliography, references, citation}}

% tables

\newcolumntype{H}{>{\sffamily\bfseries\spotcolor}l}
\newcolumntype{L}{>{\raggedright\let\\=\tabularnewline}p}
\newcolumntype{R}{>{\raggedleft\let\\=\tabularnewline}p}
\newcolumntype{C}{>{\centering\let\\=\tabularnewline}p}
\newcolumntype{V}{>{\raggedright\let\\=\tabularnewline\ttfamily}p}

\newcommand*{\sorttablesetup}{%
  \tablesetup
  \ttfamily
  \def\new{\makebox[1.25em][r]{\ensuremath\rightarrow}\,}%
  \def\alt{\par\makebox[1.25em][r]{\ensuremath\hookrightarrow}\,}%
  \def\note##1{\textrm{##1}}}

\newcommand{\tickmarkyes}{\Pisymbol{psy}{183}}
\newcommand{\tickmarkno}{\textendash}
\providecommand*{\textln}[1]{#1}
\providecommand*{\lnstyle}{}

% markup and misc

\setcounter{secnumdepth}{4}

% following snippet is based on code by Michael Ummels (TeX Stack Exchange)
% <http://tex.stackexchange.com/a/13073/8305>
\makeatletter
  \newcommand\fnurl@[1]{\footnote{\url@{#1}}}
  \DeclareRobustCommand{\fnurl}{\hyper@normalise\fnurl@}
\makeatother

\begin{document}

\printtitlepage
\tableofcontents
\listoftables

\newpage
\section{Introduction}
\label{int}
\biblatex is a bibliography system for \latex users. It surpasses the facilities provided by \bibtex and provides a very large feature set which can be somewhat overwhelming to the new user. This quick start guide aims to demonstrate the basic setup of \biblatex along with a selection of «how to» guides for common things which users typically need to do.

In what follows, references to the main \biblatex documentation are given
as <B$<$ref$>$>.

\biblatex tracks the citations made in a document and the options set to control how the citations are managed. It passes these to a backend processor \biber\footnote{At time of writing, \biblatex supports two backend processors, \biber and legacy \bibtex. New users, who are the focus audience for this guide, should use \biber.} which performs some tasks and then writes out a sorted representation of the bibliography data. \biblatex then uses this information on subsequent runs of \latex to format and print a bibliography. You can see the general workflow in figure \ref{fig1}.

See the appendix \secref{apx:hist} for some history about how \biblatex started, and how it differs from \bibtex.

\section{Getting Started}

Firstly, it is important to mention the user community \url{http://tex.stackexchange.com}. \biblatex has its own question tag, is in the top ten of active question areas and the developers visit it regularly. There is a large body of already answered questions there which will cover many of the issues which beginning users face and new questions are always welcome.

Using \biblatex is easy. Here is a basic example:

\begin{ltxexample}[style=latex]{}
\documentclass{article}

% Load the package
\usepackage{biblatex}

% Tell biblatex the name of the
% bibliography database file
\addbibresource{refs.bib}
\begin{document}

% Mention some bibliography items
% in the text ...
Someone said something interesting
in \cite{work1} and also in \cite{work2}.

% Print the bibliography
\printbibliography
\end{document}
\end{ltxexample}
%
You would run the \latex engine of choice\footnote{Typically one of \texttt{pdflatex}, \texttt{xelatex} or \texttt{lualatex}} on this file, then run \biber on the \file{.bcf} file this produces and then the \latex engine once more\footnote{Often two or more times in fact, depending on the \biblatex options chosen which may require more runs to resolve various references.}. The commands to run would look like this assuming the file above was called \file{test.tex}.

\begin{verbatim}
latex test
biber test
latex test
latex test
\end{verbatim}
%
Notice that we didn't need to specify the file extensions as \latex knows to look for the \file{.tex} file and \biber knows to look for the \file{.bcf} file. So, this is all equivalent to doing:

\begin{verbatim}
latex test.tex
biber test.bcf
latex test.tex
latex test.tex
\end{verbatim}
%
Many people will be using \latex editing applications which hide these commands from the user anyway and which will be activated by keyboard shortcuts or menu items.

One thing that \biblatex is good at is \emph{localisation}, that is, knowing what to do when  switching between languages. If using a language other than basic English, loading \sty{inputenc} (\texttt{pdflatex}) or \sty{fontspec} (\texttt{xelatex} or \texttt{lualatex}) is recommended and then one or other of the \sty{babel} or \sty{polyglossia} packages for multi-language support. \biblatex knows how to connect with \sty{babel} and \sty{polyglossia} and can use their facilities for switching language handling and even font scripts dynamically in the bibliography and in citations. For example if writing in French, using \texttt{lualatex} as the engine and \sty{babel} as the language support package, the basic document would look like this:

\begin{ltxexample}[style=latex]{}
\documentclass{article}
\usepackage[french]{babel}
\usepackage{fontspec}
\usepackage{biblatex}
\addbibresource{refs.bib}

\begin{document}

Someone said something interesting
in \cite{work1} and also in \cite{work2}.

\printbibliography
\end{document}
\end{ltxexample}
%
Language packages like \sty{babel} and \sty{polyglossia} should be loaded \emph{before} \biblatex so that it can detect which one is being used.

It is also recommended to use the \sty{csquotes} package as this integrates
with \biblatex to provide sophisticated combined citation/quotation macros
which are also multi-lingual aware. A document skeleton for biblatex use
with \texttt{xelatex} or \texttt{lualatex} might look like this:

\begin{ltxexample}[style=latex]{}
\documentclass{article}
\usepackage[english]{babel}
\usepackage{fontspec}
\usepackage{biblatex}
\usepackage{csquotes}
\addbibresource{refs.bib}

\begin{document}

Someone said something interesting
in \cite{work1} and also in \cite{work2}.

\printbibliography
\end{document}
\end{ltxexample}
%
Or with \texttt{pdflatex}, like this:

\begin{ltxexample}[style=latex]{}
\documentclass{article}
\usepackage[english]{babel}
\usepackage[T1]{fontenc}
\usepackage[utf8]{inputenc}
\usepackage{biblatex}
\addbibresource{refs.bib}

\begin{document}

Someone said something interesting
in \cite{work1} and also in \cite{work2}.

\printbibliography
\end{document}
\end{ltxexample}
%
See \msecref{int:pre} for details about required, recommended or incompatible packages. \biblatex comes with many example files which are placed in the \path{doc/latex/biblatex/examples} subdirectory of the \file{texmf} tree. These are annotated and very useful for learning how to do various things.

\section{Basics}

The \biblatex package options are detailed in the \biblatex manual (\msecref{use}).

\subsection{Choosing a style}

The first option most users will want to specify is the \opt{style} option. This decides the formatting of the bibliography and citations. There are several standard styles which \biblatex offers (\msecref{use:xbx}). 

\begin{ltxexample}[style=latex]{}
\usepackage[style=authoryear]{biblatex}
\end{ltxexample}
%
You may have separate styles for the bibliograpy and citations by using the \opt{bibstyle} and \opt{citestyle} options respectively. \opt{style} sets both \opt{bibstyle} and \opt{citestyle}:

\begin{ltxexample}[style=latex]{}
\usepackage[citestyle=authoryear, bibstyle=authortitle]{biblatex}
\end{ltxexample}

\subsection{Migrating from natbib}
\biblatex has a compatibility setting which can be used to emulate standard \sty{natbib} commands (\msecref{use:cit:nat}) and it is enabled like this:

\begin{ltxexample}[style=latex]{}
\usepackage[natbib=true]{biblatex}
\end{ltxexample}

\subsection{Sorting the bibliography}

One of the main purposes of a bibliography system is to sort the bibliography correctly. This might be by author names in a humanties paper or by a label or number for a physical sciences paper, for example. Sorting with \biblatex is completely customisable and very sophisticated, supporting full Unicode\footnote{Using an implementation of the Unicode Collation Algorithm (UCA) with Common Locale Data Repository (CLDR) tailoring.}. Sorting is done by defining a sorting «template», which is given a name. The template name is then used as the value of the sorting options. To set the default, global sorting template, use the \opt{sorting} package option:

\begin{ltxexample}[style=latex]{}
\usepackage[sorting=ynt]{biblatex}
\end{ltxexample}
%
There are several pre-defined sorting templates (\msecref{use:opt:pre:gen}). It is possible to define custom sorting templates (\msecref{aut:ctm:srt}) or override the global sorting template and use different sorting templates for different bibliography lists (\msecref{use:bib:context}). Most custom styles will automatically use a particular sorting template mandated by the style. It is also possible to customise in great detail specifically how names of authors sort and this is completely integrated with the fine-grained control users have over the construction and parsing of names. Users are not limited to Western first/middle/last style names; it is possible to customise \biblatex to comprehensively support arbitrary name parts such as patronymics, papponymics etc. See (\msecref{aut:bbx:drv}).

\subsection{Citations}

Citations are fundamental to a bibliography system. \biblatex has a rich set of such commands described in \msecref{use:cit} and also allows the user to define custom citation commands. Typically the chosen style will dictate which type of citation and therefore which citation command should be used. Commands for citing multiple works in the same citation with local or global pre/postnotes are provided. To begin with, it is recommended to try the style-independent \cmd{autocite} command and its variants as it will help to  make switching between styles easier. If something more specific is needed, look into \cmd{parencite}, \cmd{textcite} and \cmd{footcite} and their variants.

\subsection{Printing the bibliography}

The bibliography is printed by issuing the \cmd{printbibliography} command.
The options for the \cmd{printbibliography} command can be found in
\msecref{use:bib:bib}. Here are some examples of the most commonly used
options.

The title of the bibliography may be customised by changing or defining the
heading or by overriding the title which the heading uses.

\begin{ltxexample}[style=latex]{}
\printbibliography[title=References]
\end{ltxexample}
%
The general layout of the bibliography is controlled by a «bibliography environment». Custom layouts may be defined (\msecref{use:bib:hdg}) and used via the \opt{env} option:

\begin{ltxexample}[style=latex]{}
\printbibliography[env=customenv]
\end{ltxexample}
%
Bibliographies may be filtered and split so that, for example, it is possible to have separate bibliographies for books vs articles, primary vs secondary sources etc. The filtering can be done using various criteria and complex custom filters can be defined (\msecref{use:bib:flt}). There are several interfaces for bibliography filtering, depending on the use-case. See \msecref{use:use:mlt} for examples.

\section{Delving Deeper}

\biblatex has a \emph{lot} of options and is extremely customisable. The
reference documentation for \biblatex is 300 pages which seems daunting for
a new user but in fact, the plethora of options has developed to
accommodate a very diverse array of requirements, most of which any one
individual user will never need to use. So, the rest of this guide is
structured as a list of useful tasks and options with pointers to the main
documentation where you will find comprehensive documentation and examples.
In most \TeX distributions, you can type \cmd{texdoc biblatex} at a command
prompt to view the main \biblatex PDF manual.

Please note that \biber has its own PDF documentation which
is useful if you need to know about \biber specific things like the format
of the XML-based \biber configuration file. With most \TeX distributions,
you should be able to type \cmd{texdoc biber} at a command prompt to view
the \biber manual. Running \cmd{biber --help} from the command line is the
best source of information about the options which \biber supports.

In what follows, \biber is presupposed as the backend processor as
explained above. No example are given as the main documentation has many
examples--this guide is intended as a quick pointer into the larger
documentation file.
\subsection{How do I ...?}

\subsubsection{Use multiple bibliography data sources?\msecpref{use:bib:res}}

Bibliographic data can be kept in any number of files (commonly,
\file{.bib} files) and as many as you like can be used. Full BSD0-style
file pattern globs are recognised.
  
\subsubsection{Use remote data sources?\msecpref{use:bib:res}}

Remote datasources can be fetched via HTTP(S) or FTP(S). The remote source
only has to return text of the expected format (\file{.bib} for
example)--this might be a literal web page or a web service returning the
data.

\subsubsection{Manipulate the bibliographic data without changing the
  source files?\msecpref{aut:ctm:map}}

It is possible to change source data in almost arbitrary ways as it is read,
without modifying the data files. This is useful, for example for
normalisation and also in situations where you do not control the data
files. This is done by <source mapping> which \biber applies as it reads
the data files. Source maps are defined with the \cmd{DeclareSourcemap}
macro. Many data manipulation operations are possible--regular expression
matching and replace, entry and field cloning, deletion of entries or
fields, appending to fields etc.

\end{ltxexample}

\subsubsection{Customise Sorting the Bibliography?\msecpref{aut:ctm:srt}}

Sorting is one of the main strengths of \biber and uses a full Unicode
engine with tailoring for specific languages. \biblatex comes with
several pre-defined sorting templates to sort the bibliography in standard
ways\msecpref{use:opt:pre:gen} but users can define their own sorting
templates with the \cmd{DeclareSortingTemplate} command. Sorting is
completely customisable and it is possible to sort on any fields, in any
order with a choice, per-field, of ascending/descending, case-sensitiviy,
language locale etc. It is also possible to ignore specific parts of any data
for sorting purposes (for example <The> in titles or particular diacritics
preceding names in some languages)\msecpref{aut:ctm:nosort}.

\subsubsection{Truncate long name lists?\msecpref{use:opt:pre:gen}}

See the \opt{maxnames/minnames} options and their sub-options. Name lists
can be explicitly truncated with <et al.> to a desired explicit length.
There is also a powerful automated option \opt{uniquelist} which will
automatically truncate name lists to the point of no ambiguity.

\subsubsection{Disambiguate ambiguous names?\msecpref{use:opt:pre:gen}}

See the \opt{uniquename} option. This has various settings to allow the
user to choose how potentially ambiguous names in a name list are
disambiguated. Names can be disambiguated by adding initials or the full
text of other nameparts to the basic name part (usually the family name) in
order to disambiguate names. 

\subsubsection{Handle names for non-Western languages?\msecpref{aut:bbx:drv}}

See the \opt{nameparts} constant. By default, this adheres to the standard
\bibtex name model of <prefix, family, suffix, given> but this can be
customised to support arbitrary name parts such as patronymics,
papponymics etc. \biber supports an extended name format in \bibtex data
sources which allows names to be specified in a completely extensible way
and is not limited to the restrictive \bibtex model. \biblatex takes care
of automatically creating all of the formatting commands required to work
with any nameparts defined using the \opt{nameparts} constant. The main
documentation contains references to example files (which come with
\biblatex) demonstrating all aspects of customised name handling.

\subsubsection{Handle date and time information?\msecpref{bib:use:dat}}

\biblatex has comprehensive support for \acr{ISO8601-2} Extended Format
specification level 1 date/times in the bibliography. Input and
customisable output for unspecified and uncertain dates, Julian dates,
BCE/BC eras, approximate dates, negative dates and data ranges are fully
supported.

\subsubsection{Handle crossreferences and data inheritance between entries?\msecpref{bib:cav:ref}}

Entries can reference other entries and inherit data from them in a
completely customisable and cascading manner. See the
\cmd{DeclareDataInheritance} command\msecpref{bib:cav:ref}.

\subsubsection{Deal with different file encodings?\msecpref{bib:cav:enc:enc}}

\biber is completely UTF-8 internally. It decodes and encodes in
input/ouput as required and will automatically convert \tex macros in the
bibliographic data into UTF-8 on reading data to aid sorting and can
re-encode to arbitrary encodings on output.


% \begin{lstlisting}[style=bibtex]{}
% \end{lstlisting}

% \begin{ltxexample}[style=latex]{}
% \end{ltxexample}


\newpage
\appendix
\section*{Appendix}
\addcontentsline{toc}{section}{Appendix}

\section{History}
\label{apx:hist}
\biblatex was developed in order to allow bibliography styles to be created using \TeX\ macros, which apply to a \file{.bbl} file treated as a sorted database rather than as a typset bibliography. \biblatex does this by using only one special \bibtex \file{.bst} «style» file which describes the special \file{.bbl} file format. 

In a sense, \biblatex abused \bibtex for its data reading and sorting functions while ignoring arguably its main function of formatting bibliographic data. In 2008, development on \biber started which aimed to provide a dedicated and customised backend for \biblatex to remove the dependency on \bibtex. The goals were:

\begin{itemize}
\item Full Unicode support, which is lacking in all \bibtex (the program) variants to some degree
\item Better, more flexible sorting algorithm
\item Customisable interface file format to allow far more options than are possible with \bibtex (the progam)
\end{itemize}
%
\biber now fulfills these goals along with a large additional feature set. \biblatex with \biber still supports (and always will) the \bibtex file format (\file{.bib}) as this is the most widely used format in the \latex world. However, it is not limited to this and supports other data formats with a modular internal design to allow relatively easy addition of other data formats.

\biber takes over all of the tasks previously done by \bibtex for \biblatex. One change is that \biber does not read the \file{.aux} file but reads a more complex and structured file which \biblatex uses to pass information to \biber. This is the XML format \file{.bcf}, the Biblatex Control File. See Figure \ref{fig1} for an overview.

\begin{figure}
  \centering
  \includegraphics[width=\textwidth,keepaspectratio=true]{qs1.png}
  \caption{biblatex workflow}
  \label{fig1}
\end{figure}

\end{document}
%%% Local Variables:
%%% coding: utf-8
%%% eval: (visual-line-mode 1)
%%% eval: (auto-fill-mode -1)
%%% End:
